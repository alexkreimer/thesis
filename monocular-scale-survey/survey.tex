\documentclass{article}

\usepackage[utf8]{inputenc}
\usepackage{mathtools, amssymb}
\usepackage{subcaption}
\usepackage{graphicx}
\usepackage{booktabs}

\graphicspath{ {./images/} }

\DeclarePairedDelimiter{\abs}{\lvert}{\rvert}
\DeclarePairedDelimiter{\norm}{\lVert}{\rVert}

\title{Literature Survey: Scale Estimation in the Monocular SLAM}
\author{Alex Kreimer}
\date{May 2016}

\begin{document}

\maketitle

\section{Absolute Scale in Structure from Motion from a Single Vehicle
  Mounted Camera by Exploiting Nonholonomic
  Constraints~\cite{scaramuzza2009absolute}}

The authors propose a method to estimate the absolute scale of the
camera motion for a single camera.  The method exploits non-holomicity
of the car motion (e.g. the presence of the Instantaneous Center of
Rotation during the turns.) This will work for every robot that
adheres to Ackerman steering principle, and will not work for other
types of vehicles). The camera needs to be offset from the
non-steering axis of the car and then it is possible to exactly derive
the scale of the motion from the system geometry and the the camera
odometry estimation.

\section{Reliable Scale Estimation and Correction for Monocular Visual Odometry}

The authors estimate ground plane homography that related subsequent
images.  The feature points are taken from an a-priory selected ROI.
The contribution of the paper is to separate motion parameters and
plane parameters.  The camera motion is estimated from all the feature
points available in the image (not just the ROI), which makes it more
robust.  Subsequently this estimate is used to compute the geometry of
the ground plane from the homography (e.g. $H = K(R-tn^T/d)K^{-1}$)

\bibliography{survey}{} \bibliographystyle{plain}
\end{document}

%%% Local Variables:
%%% mode: latex
%%% TeX-master: t
%%% End:
