\chapter{Conclusion}

This work treats two related sub problems in the field of vision aided
navigation.

The first part is of geometric nature and attempts to provide a new
insight into stereo and monocular motion estimation.  We show that if
one is able to split the image points into disjoint sets of near and
distant points, then we provide an estimation procedure for camera
motion.  Our algorithm is local by design and compares favorably to
its baseline.  By local we mean, that it does not build a map of the
environment and thus all its decisions are based on a pair of frames
that it sees.  Of course, there are larger SLAM systems that give
better accuracy overall.  Our algorithm may be used as a building
block of such systems or may be applicable in the compute-restrained
environment.

The second part of our work takes a statistical machine learning
approach, which leverages available data and compute power to solve
the visual navigation scaling problem.  We show how to train a model
to provide unbiased and low-variance scale estimates.  We also show a
simple approach which uses our scale estimates to provide full 6-DOF
motion estimation.  The work of~\cite{frost2017using} show a
different, more complex way to do motion estimation based on similar
scale estimates.  While the current result allows for scale drift free
motion estimation we feel that it may be greatly improved by better
use of the available data.

In recent years the field of visual navigation has matured
research-wise. It seems that most of the important fundamental
questions are answered.  Said that, there still is a significant gap
between state of the art and real world systems that rely on visual
data.  Part of this gap is in engineering, which is complex (most real
world systems use additional modalities: lidars, depth sensors,
etc.). We speculate that another part of the gap is that in order to
navigate in complex (e.g., urban) environments one can not rely on
geometric information alone and semantic understanding of the scene is
needed.

It also seems that the field is ready to take on larger-scale
problems, such as motion estimation and mapping on a very large scale,
collaborative large-scale mapping and navigation.  We feel that (at
least near-future) algorithms should leverage the power of traditional
geometry based approaches with a more recent data-driven methods.

One thing that we find very important is the appearance of new and
more challenging data sets.  KITTI, while being a great data set is
low in its lighting, weather conditions, scenery and traffic
variability.  We hope that the recent spree of the autonomous driving
companies will lead to the appearance of such new data sets, which in
turn, will foster new research.

%%% Local Variables:
%%% mode: latex
%%% TeX-master: t
%%% End:
